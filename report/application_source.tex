






\section{Ο κώδικας του λογισμικού}

Όπως αναφέρθηκε στην εισαγωγή το παρόν πρόγραμμα έχει γραφεί σε C++. Σκοπός
είναι η δημιουργία μιας desktop εφαρμογής σε Qt6 περιβάλλον η οποία δεν
θα υλοποιεί τις ανάγκες της εφαρμογής και θα αξιοποιεί τους πόρους του
συστήματος για καλύτερες αποδόσεις. Στην συνέχεια θα περιγραφεί ο τρόπος
με τον οποίο λειτουργεί η εφαρμογή αναφέροντας τις σημαντικότερες διαδικασίες
που αναγράφονται σε κώδικα και εξηγούν πως φτάνουμε στο τελικό αποτέλεσμα
που είδαμε προηγουμένως.


\section{Η συνάρτηση main}
Κάθε πρόγραμμα γραμμένο σε C++ ξεκινάει με την συνάρτηση main να αποτελεί το
σημείο εκκίνησης του προγράμματος. Επομένως αξίζει να δούμε αναλυτικότερα τι
συμβαίνει σε αυτό το κομμάτι του κώδικα για να μπορέσουμε να κατανοήσουμε
καλύτερα την ροή και τις ενέργειες του προγράμματος.

\begin{lstlisting}[language=C++, style=cppstyle]
int main(int argc, char* argv[]) 
{
    QApplication app(argc, argv);
    
    ApplicationExplorer explorer(app);
    explorer.setWindowTitle("P2019140 - Konstantinos Tourtsakis");
    explorer.setWindowFlags(Qt::WindowCloseButtonHint | Qt::FramelessWindowHint);
    explorer.resize(800, 600);
    
    // Storing screen resolution
    screen_width = GetSystemMetrics(SM_CXSCREEN);
    screen_height = GetSystemMetrics(SM_CYSCREEN);
    
    explorer.CreateUI();
    
    VirtualKeyboard QKeyboard;
    
    //QKeyboard.setWindowFlags(Qt::WindowCloseButtonHint | Qt::FramelessWindowHint);
    explorer.QKeyboard = &QKeyboard;
    

    explorer.showMaximized();
    return app.exec();
}
\end{lstlisting}

Αρχικά δημιουργείται το αντικείμενο της εφαρμογής που δέχεται τα arguments με τα οποία εκτελέστηκε.
Στην συνέχεια δημιουργείται ένα αντικείμενο τύπου ApplicationExplorer. Το αντικείμενο αυτό αποτελεί
το βασικό παράθυρο της εφαρμογής το οποίο κληρονομεί την κλάση QWidget και χτίζει τα αντικείμενα της
γραφικής διεπαφής. Στην συνέχεια προστίθενται ορισμένα flags στο παράθυρο της εφαρμογής. Μετά αποθηκεύονται
οι διαστάσεις της οθόνης του χρήστη σε global μεταβλητές οι οποίες θα χρησιμεύσουν στην προσαρμογή των
διαστάσεων των στοχείων της γραφικής διεπαφής ανάλογα με τις διαστάσεις της οθόνης. Μετά ακολουθεί η
δημιουργία των γραφικών στοιχείων καλώντας την μέθοδο CreateUI. Η εφαμοργή συνεχίζει με την δημιουργία
ενός ακόμη αντικειμένου. Αυτήν την φορά η κλάση QWidget κληρονομείται από την κλάση VirtualKeyboard
η οποία δημιουργεί και διαχειρίζεται το παράθυρο του εικονικού πληκτρολογίου. Στην συνέχεια δημιουργείται
αντίγραφο της θέσης μνήμης του αντικειμένου αυτού σε μια μεταβλητή της κλάσης του αρχικού παραθύρου με
σκοπό την επικοινωνία μεταξύ των δύο παραθύρων της εφαρμογής κατά την περιήγηση του χρήστη. Τέλος γίνεται
κλήση για προβολή του βασικού παραθύρου της εφαρμογής με τις μέγιστες διαστάσεις παραθύρου και μετά
εκτελείται η επανάληψη της Qt6 εφαρμογής. 


\section{Η κλάση του προγράμματος}


Μια εφαρμογή που αξιοποιεί τo Qt6 περιλαμβάνει αντικείμενα διαφόρων τύπων ώστε να
μπορέσει να λειτουργήσει. Όπως τα περισσότερα στοιχεία της εφαρμογής, έτσι και το
παράθυρο που δημιουργείται για να φιλοξενήσει την γραφική διεπαφή της είναι ένα αντικείμενο. 
Στην προκειμένη περίπτωση κληρονομείται η κλάση QWidget και στην συνέχεια χρησιμοποιείται 
ως βάση για την υλοποίηση της εφαρμογής. Παρακάτω βλέπουμε τον ορισμό των χαρακτηριστικών
της κλάσης και 


\begin{lstlisting}[language=C++, style=cppstyle]

\end{lstlisting}





\subsection{Ο constructor}
Παρακάτω βλέπουμε την υλοποίηση του constructor της εφαρμογής στον οποίο δημιουργείται 
το χρονόμετρο με το οποίο εκτελείται η διεργασία περιήγησης του χειριστηρίου βιντεοπαιχνιδιών.

\begin{lstlisting}[language=C++, style=cppstyle]
class ApplicationExplorer : public QWidget
{
    ApplicationExplorer(QApplication& app, QWidget* parent = nullptr) : QWidget(parent), app(app)
    {
        // Controller task loop - Perform task constantly
        timer = new QTimer(this);
        connect(timer, &QTimer::timeout, this, &ApplicationExplorer::TaskGamepadNavigation);
        timer->start();

    }
};
\end{lstlisting}



\subsection{Η μέθοδος SetupUI}
\subsection{Η μέθοδος UpdateListWidget}
\subsection{Η μέθοδος ShowCustomContextMenu}
\subsection{Η μέθοδος SetupUI}
\subsection{Η μέθοδος SetupUI}



\begin{lstlisting}[language=C++, style=cppstyle]
\end{lstlisting}