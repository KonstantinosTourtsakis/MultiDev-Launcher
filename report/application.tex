





\section{Περιγραφή λογισμικού}


Το λογισμικό αυτό αποτελείται από 3 μέρη τα οποία ουσιαστικά αποτελούν και
τις περιφερειακές συσκευές τις οποίες υποστηρίζει για την επίλυση του προβλήματος.
Αυτά είναι τα εξής:

\section{Εκτέλεση εφαρμογών με το ποντίκι}


Το ποντίκι όπως είναι γνωστό είναι ένα εξαιρετικά απλό μέσο με το οποίο επικοινωνεί
ο χρήστης με τον υπολογιστή και επιτρέπει την χρήση ενός μόνο χεριού για την περιήγηση
μέσα σε αυτόν. Επομένως, η υποστήριξη της συσκευής αυτής από την εφαρμογή γίνεται
μέσω της γραφικής διεπαφής από όπου και είναι εύκολα προσβάσιμα τα στοιχεία που
διαχειρίζονται την εκτέλεση, προσαρμογή και διαχείρηση των προγραμμάτων που είναι
εγκατεστημένα στο περιβάλλον του χρήστη. Για τον σκοπό αυτό έχουν προστεθεί τα κατάλληλα
στοιχεία γραφικής διεπαφής που υλοποιούν την προβλεπόμενη από τον τελικό χρήστη 
λειτουργικότητα της εφαρμογής. Πιο συγκεκριμένα, η γραφική διεπαφή έχει χωριστεί σε τρία κομμάτια.
Αυτά υλοποιούνται μέσω του Qt σε καρτέλες οι οποίες στην συνέχεια προστίθενται στο
κεντρικό QTabWidget που αποτελεί κομμάτι τoυ βασικού γραφικού στρώματος του προγράμματος.
Για τις ανάγκες της εφαρμογής έχει δημιουργηθεί μια καρτέλα με τις εφαρμογές που γνωρίζει
η εφαρμογή, μια καρτέλα με τις αγαπημένες εφαρμογές του χρήστη και μια καρτέλα με τις
ρυθμίσεις που μπορεί να αλλάξει ο χρήστης για να φέρει την λειτουργικότητα του προγράμματος
πιο κοντά με τις προσωπικές του προτιμήσεις.

\begin{lstlisting}[language=C++, style=cppstyle]
tabs->addTab(tab_all_apps, "Applications");
tabs->addTab(tab_favorites, "Favorites");
tabs->addTab(tab_settings, "Settings");

layout_root->addWidget(tabs);
\end{lstlisting}

Ας αναλύσουμε τώρα την κάθε καρτέλα ξεχωριστά ξεκινώντας από την καρτέλα των ρυθμίσεων για
λόγους συνοχής. Πρώτα στην κα







\begin{lstlisting}[language=C++, style=cppstyle]
\end{lstlisting}