\documentclass[a4paper,11pt,oneside,openany]{ioniothesis}



\usepackage[dvips]{graphicx}
\usepackage{makeidx}
\usepackage[greek, english]{babel}
%\usepackage[iso-8859-7]{inputenc}
\usepackage{fontspec}
\setmainfont{Arial}
\setmonofont{Courier New}
\usepackage{amssymb}
\usepackage{amsfonts}
\usepackage{color}
\usepackage{ioniostyle}
\usepackage{xcolor}
\usepackage{listings}
\usepackage{enumitem}

\selectlanguage{greek}


\def\tl{\textlatin}
\def\tg{\textgreek}


 
\parindent=0pt

\makeindex
\evensidemargin=0 cm
\oddsidemargin=1 cm
\textwidth=15 cm







% Define custom colors
\definecolor{codebg}{RGB}{240,240,240}
\definecolor{termfg}{RGB}{0,255,255}
\definecolor{termcodebg}{RGB}{40,42,54}



% Define code listing style for C++
\lstdefinestyle{cppstyle}{
    backgroundcolor=\color{codebg},
    basicstyle=\footnotesize\ttfamily,
    keywordstyle=\color{blue},
    stringstyle=\color{red},
    commentstyle=\color{green!50!black},
    morecomment=[l][\color{magenta}]{\#}
}





\newtheorem{theo}{Θεώρημα}[chapter]
\newtheorem{define}{Ορισμός}[chapter]
\newtheorem{collary}{Πόρισμα}[chapter]
\newtheorem{algorithm}{Αλγόριθμος}[chapter]
\newtheorem{example}{Παράδειγμα}[chapter]




%The following line is for one and half spacing
%\linespread{1.3}
%The following line is for double spacing
%\linespread{1.6}

\linespread{1.5}

\usepackage{drop}


\begin{document}




\author{\textbf{Κωνσταντίνος Τουρτσάκης, Π2019140} \\ \textbf{Επιβλέπων:}  -- Μιχαήλ Στεφανιδάκης --}
\title{
\textbf{\LARGE{\textsc{Ιόνιο Πανεπιστήμιο}} 
\bigskip \\
\large{\textsc{Τμήμα Πληροφορικής}}
\bigskip \\ \bigskip \bigskip \bigskip
\bigskip \bigskip 
\includegraphics[width=0.2\textwidth]{./pdffigs/ionio_logo.pdf}
\bigskip \\ \bigskip 
%\texttt{\large{-- Πτυχιακή Εργασία --}}
\texttt{-- Πτυχιακή Εργασία --}
\bigskip \\ \bigskip
\textbf{\Large{\texttt{
Διεπαφή χρήστη διαχείρισης και εκτέλεσης \\
εφαρμογών μέσω πολυτροπικών \\
περιφερειακών συσκευών
}}}
\bigskip \\ \bigskip}
}
\maketitle





\chapter*{}


\begin{center}
\Large{\textbf{Επιβλέπων}}
\end{center}
\prof{Μιχαήλ Στεφανιδάκης}{Αναπληρωτής Καθηγητής}{Ιόνιο Πανεπιστήμιο Τμήμα Πληροφορικής}

\begin{center}
\Large{\textbf{Τριμελής Επιτροπή}}
\end{center}
\prof{Μιχαήλ Στεφανιδάκης}{Αναπληρωτής Καθηγητής}{Ιόνιο Πανεπιστήμιο Τμήμα Πληροφορικής}
\prof{Δημήτριος Ρίγγας}{ΕΔΙΠ}{Ιόνιο Πανεπιστήμιο Τμήμα Πληροφορικής}
\prof{Θεόδωρος Ανδρόνικος}{Αναπληρωτής Καθηγητής}{Ιόνιο Πανεπιστήμιο Τμήμα Πληροφορικής}







  
\pagenumbering{roman}


%Περίληψη
\chapter*{Περίληψη} \pagestyle{headings}







Η χρήση του προσωπικού υπολογιστή έχει παραμείνει η ίδια εδώ και δεκάδες χρόνια.
Σε αντίθεσή όμως με τις εποχές που οι υπολογιστές αποτελούσαν συσκευές που λίγοι
είχαν στην διάθεσή τους, πλέον υπάρχουν σε ένα μεγάλο ποσοστό στα σπίτια των
ανθρώπων. Η χρήση για την οποία προορίζεται επομένως έχει διαφοροποιηθεί και
για αυτόν τον λόγο είναι αναγκαίο να μπορεί ο χρήστης να περιηγηθεί μέσα στο σύστημα
ανάλογα με τις συσκευές εισόδου που ικανοποιούν και αιτιολογούν την χρήση αυτή.
Το πρόγραμμα αυτό προσπαθεί να λύσει αυτό το πρόβλημα, προσθέτοντας λειτουργικότητα
με την οποία ο χρήστης θα μπορεί να περιηγηθεί χρησιμοποιώντας είτε το πληκτρολόγιο,
είτε το ποντίκι είτε ένα χειριστήριο κονσόλας βιντεοπαιχνιδιών. Για την υλοποίηση του
προγράμματος αυτού θα αξιοποιηθεί η βιβλιοθήκη γραφικής διεπαφής Qt, η βιβλιοθήκη
XInput με το οποίο γίνεται διαχείριση εντολών εισόδου χειριστηρίων Xbox και το
σύστημα Windows στο οποίο θα εστιάσει η εφαρμογή. Το Qt είναι μια cross-platform βιβλιοθήκη
για ανάπτυξη εφαρμογών γραφικής διεπαφής και παρέχει οτιδήποτε χρειάζεται μια τέτοιου είδους εφαρμογή. Ο χρήστης κατά την εισαγωγή του
στο πρόγραμμα καλείται να εισάγει το όνομα χρήστη που θα χρησιμοποιήσει η εφαρμογή.
Στην συνέχεια εισέρχεται στο βασικό παράθυρο της εφαρμογής το οποίο αξιοποιείται κυρίως
από το ποντίκι και  το χειριστήριο. Το παράθυρο αυτό έχει 3 καρτέλες. Μια με τις εφαρμογές
του χρήστη, μία με τις αγαπημένες εφαρμογές του χρήστη και μία με τις ρυθμίσεις της εφαρμογής
στις οποίες μπορεί να την παραμετροποιήσει . Στις ρυθμίσεις μπορεί επίσης να βρει και ένα
κουμπί για το άνοιγμα του παραθύρου ενός εικονικού πληκτρολογίου στο οποίο έχει πρόσβαση και
μέσω του χειριστηρίου βιντεοπαιχνιδιών. Επιπλέον υπάρχει ένα παράθυρο για την εκτέλεση 
εφαρμογών μέσω του πληκτρολογίου. Αυτό το παράθυρο διευκολύνει την εκτέλεση τους μέσω της
συσκευής αυτής προσθέτοντας λειτουργικότητα περιήγησης πάνω σε αυτό που αντιμετωπίζει προβλήματα
όπως εστίαση σε λάθος αντικείμενα πάνω στην γραφική διεπαφή. Στον κώδικα της εφαρμογής υπάρχει
μια κλάση στην οποία δημιουργείται η γραφική διεπαφή του προγράμματος αυτού, μια κλάση για την
δημιουργία του εικονικού πληκτρολογίου και μια κλάση για τον έλεγχο εντολών εισόδου από το
χειριστήριο βιντεοπαιχνιδιών. Επιπλέον υπάρχουν συναρτήσεις που "προσομοιώνουν" πατήματα πλήκτρων
και χρησιμεύουν στην αντιστοίχηση των πιέσεων των κουμπιών του χειριστηρίου σε συντομεύσεις που
υπάρχουν στο σύστημα. Τέλος, η εφαρμογή αυτή μπορεί να εγκατασταθεί μέσω ενός installer που έχει
δημιουργηθεί για τον σκοπό αυτό. Ο installer αυτός είναι διαθέσιμος στο αντίστοιχο αποθετήριο που
υπάρχει στο GitHub και περιλαμβάνει τον πηγαίο κώδικα της εργασίας.


\cleardoublepage

%Πρόλογος και Ευχαριστίες
%\chapter*{Πρόλογος και Ευχαριστίες} \pagestyle{headings}
%\input{prologos}


\cleardoublepage

\tableofcontents
\cleardoublepage




\listoffigures
\cleardoublepage
\listoftables

\setlength{\parskip}{5pt}



\pagestyle{headings}
\cleardoublepage


\newpage
\pagenumbering{arabic}


\cleardoublepage


\chapter{Εισαγωγή} \label{chapter:intro}




\drop{Ε}{ίναι}
πλέον σίγουρο πως η χρήση ενός προσωπικού υπολογιστή γίνεται μέσω του
πληκτρολογίου σε συνδυασμό με την χρήση του ποντικιού για την περιήγηση του
χρήστη μέσα στο γραφικό περιβάλλον των σύγχρονων συστημάτων. Όμως αυτό ήταν
ανέκαθεν μια μέθοδος περιήγησης που προορίζονταν για εργασιακή χρήση. Με την
εξέλιξη των τεχνολογιών και την εισαγωγή ολοένα και περισσότερων πολυμεσικών
εφαρμογών στο περιβάλλον του Η/Υ, ήταν αναπόφευκτη η μετάβαση σε μια εποχή
όπου ο Η/Υ έχει εφαρμογή σε κάθε σπίτι ανεξαρτήτως του επαγγέλματος του
ιδιοκτήτη. Αναγνωρίζοντας την αλλαγή αυτή και το γεγονός πως παραμένουν
περιθώρια βελτίωσης από την πλευρά του συστήματος ώς προς την επικοινωνία
μεταξύ του χρήστη και του υπολογιστή, το πρόγραμμα αυτό έχει ως στόχο την
εκτέλεση εφαρμογών ανεξαρτήτως της συσκευής εισόδου του χρήστη. Επομένως,
απώτερος σκοπός είναι η αξιοποίηση των νέων περιφερειακών συσκευών στην
εκτέλεση και περιήγηση του λειτουργικού συστήματος αλλά και την υλοποίηση
νέων διαδικασιών εκτέλεσης εφαρμογών από προϋπάρχουσες συσκευές, όπως το
πληκτρολόγιο και το ποντίκι. Για την υλοποίηση αυτής της εφαρμογής είναι
απαραίτητο να γίνει χρήση τεχνολογιών για την γραφική διεπαφή χρήστη και την
είσοδο δεδομένων από τις περιφερειακές συσκευές στο σύστημα. Για την
εξυπηρέτηση αυτών των αναγκών γίνεται κυρίως χρήση της βιβλιοθήκης Qt6, του
XInput και της βιβλιοθήκης των Windows.








\chapter{Ανάπτυξη εφαρμογών σε Qt6} \label{chapter:qt6}





%\section{Qt6 Desktop Applications}
\section{Τι είναι το Qt;}
Το Qt είναι μια δημοφιλής βιβλιοθήκη ανάπτυξης εφαρμογών διεπαφής χρήστη το
οποίο είναι διαθέσιμο σε C++ και Python. Με το Qt
είναι εφικτή η υλοποίηση cross-platform εφαρμογών με το τελικό αποτέλεσμα
να είναι μια αξιοπρεπής διεπαφή χρήστη που λειτουργεί αποτελεσματικά και
αξιόπιστα, υποστηρίζοντας μια πληθώρα συστημάτων όπως Windows, Linux, MacOS, Android και ενσωματωμένα συστήματα. 
Παρακάτω γίνεται περιγραφή της λειτουργίας της βιβλιοθήκης αυτής όπου στην 
περίπτωση αυτή θα γίνει χρήση παραδειγμάτων σε γλώσσα προγραμματισμού C++ 
μιας και είναι η γλώσσα στην οποία έχει γραφεί το παρόν πρόγραμμα.

\section{Ένα απλό πρόγραμμα σε Qt6}
Για την δημιουργία ενός παραθύρου Qt6 πρέπει πρώτα να δημιουργηθεί ένα αντικείμενο
τύπου \textbf{QApplication} και στην συνέχεια να αρχικοποιηθεί ένα αντικείμενο 
\textbf{QWidget} το οποίο θα αποτελεί το παράθυρο της εφαρμογής πάνω στο
οποίο θα προστεθούν τα υπόλοιπα στοιχεία της γραφικής διεπαφής για τις λειτουργίες
του προγράμματος. Παρακάτω βλέπουμε ένα παράδειγμα με την αντίστοιχη περιγραφή
μεταφρασμένη σε κώδικα.

\begin{lstlisting}[language=C++, style=cppstyle]
#include <QApplication>
#include <QWidget>


class MyWidget : public QWidget 
{
public:
    MyWidget(QWidget *parent = nullptr) : QWidget(parent) 
    {
        setFixedSize(400, 300);
        setWindowTitle("P2019140 - Konstantinos Tourtsakis");
    }
};

int main(int argc, char *argv[]) 
{
    QApplication app(argc, argv);

    MyWidget widget;
    widget.show();

    return app.exec();
}

\end{lstlisting}

\includegraphics[width=1.0\textwidth]{./images/simple_qt6_app.png}

Η κλάση MyWidget κληρονομεί την κλάση QWidget. Το παράθυρο το οποίο δημιουργείται
εμφανίζεται σε πλήρες μέγεθος και στην συνέχεια επιστρέφεται το αντικείμενο της εφαρμογής
το οποίο το διαχειρίζεται η βιβλιοθήκη κατά την έξοδο εκτέλεσης του προγράμματος.


\section{Προσθήκη στοιχείων γραφικής διεπαφής σε Qt6}
Κάθε εφαρμογή γραφικής διεπαφής παρέχει στοιχεία μέσω των οποίων γίνεται η διαχείριση
των δεδομένων που επεξεργάζεται και η εκτέλεση των λειτουργιών του. Συνήθως τα δεδομένα αυτά δεν είναι τίποτε άλλο από
τους βασικούς τύπους δεδομένων που υποστηρίζουν οι γλώσσες προγραμματισμού:
int, bool, float και string. Επιπλέον υπάρχουν στοιχεία με τα οποία εξυπηρετήται
αποτελεσματικότερα ο σκοπός του προγράμματος, είτε λόγω ευκολίας είτε λόγω κατανόησης
από τον χρήστη. Παρακάτω βλέπουμε τα στοιχεία που αξιποιεί το πρόγραμμα της
εργασίας για την επίτευξη του σκοπού του.

\subsection{Δημιουργία QPushButton}
Ένα QPushButton είναι ένα κουμπί το οποίο έχει την ιδιότητα εκτέλεσης λειτουργιών.
Για την προσθήκη μιας λειτουργίας πάνω στο κουμπί αυτό χρειάζεται να γίνει η
σύνδεση μεταξύ του αντικειμένου αυτού και μιας μεθόδου που ανήκει στην κλάση που
κληρονομεί το QWidget. Ορίζεται το signal το οποίο υποστηρίζει το κάθε αντικείμενο
μέσω του οποίου θα γίνει η κλήση της μεθόδου που ονομάζεται slot από το Qt.
Τα signals που υποστηρίζουν τα στοιχεία του Qt είναι διαφορετικά, ανάλογα με τους
στόχους που προσπαθεί να πετύχει το κάθε ένα από αυτά. Επομένως ένα στοιχείο QPushButton
μπορεί να έχει περισσότερα ή λιγότερα singals σε σύγκριση με ένα QComboBox.
Παρακάτω βλέπουμε ένα παράδειγμα σε κώδικα.
\begin{lstlisting}[language=C++, style=cppstyle]
class MyWidget : public QWidget 
{
public:
    MyWidget(QWidget *parent = nullptr) : QWidget(parent) 
    {
        setFixedSize(300, 200);
        setWindowTitle("P2019140 - Konstantinos Tourtsakis");

        QPushButton *button = new QPushButton("Hello, World", this);
        button->setGeometry(100, 100, 100, 30);

        connect(button, &QPushButton::clicked, this, &MyWidget::PrintHello);
    }

public slots:
    void PrintHello() 
    {
        std::cout << "Hello, World!" << std::endl;
    }
};
\end{lstlisting}

\includegraphics[width=1.0\textwidth]{./images/QPushButton.png}

Στο παράδειγμα γίνεται ή δημιουργία και ή αρχικοποίηση κουμπιού με το όνομά του και στην συνέχεια η σύνδεση.
Στην σύνδεση καλείται η μέθοδος \textbf{connect} στην οποία ορίζεται το signal το οποίο
θα πυροδοτήσει την κλήση του slot που έχει ανατεθεί στο αντικείμενο. Στην προκειμένη περίπτωση έχουμε
ορίσει το QPushButton::clicked signal το οποίο συνδέει το κουμπί με την μέθοδο PrintHello.

\subsection{Δημιουργία γραφικού πλαισίου}
Ένα layout είναι ένα πλαίσιο στο οποίο μπορούν να τοποθετηθούν άλλα στοιχεία του Qt, όπως
το QPushButton που προαναφέρθηκε. Το Qt6 παρέχει 3 βασικά είδη layouts. Το \textbf{QVBoxLayout}, το
\textbf{QHBoxLayout} και το \textbf{QGridLayout}. Τα πρώτα δύο παρέχουν ένα παρόμοιο πλαίσιο με την μόνη
τους διαφορά να βρίσκεται στην κατεύθυνση των στοιχείων μέσα στο πλαίσιο. Επομένως, ένα
QVBoxLayout χρησιμοποιείται για στοιχεία που θα τοποθετηθούν κάθετα (vertical) και ένα
QHBoxLayout θα χρησιμοποιηθεί για στοιχεία που πρόκειται να τοποθετηθούν όριζόντια (horizontal).
Τέλος, ένα QGridLayout χρησιμοποιείται για την τοποθέτηση στοιχείων σε μορφή πίνακα. Το
πλαίσιο αυτό διαθέτει θέσεις που αναπαριστούν ένα σημείο σε έναν δισδιάστατο χώρο. Κάθε σημείο
έχει μια θέση η οποία είναι μοναδική και είναι προσβάσιμη μέσω της τιμής της σειράς και της
στήλης στην οποία βρίσκεται. Παρακάτω βλέπουμε κώδικα με την χρήση ενός QVBoxLayout και την
προσθήκη ενός QPushButton σε αυτό.
\begin{lstlisting}[language=C++, style=cppstyle]
	MyWidget(QWidget *parent = nullptr) : QWidget(parent) 
    {
        setFixedSize(300, 200);
        setWindowTitle("P2019140 - Konstantinos Tourtsakis");
        
        QVBoxLayout *layout = new QVBoxLayout(this);

        QPushButton *button = new QPushButton("Hello, World", this);
        button->setGeometry(100, 100, 100, 30);

        layout->addWidget(button);

        connect(button, &QPushButton::clicked, this, &MyWidget::PrintHello);
    }

\end{lstlisting}

\includegraphics[width=1.0\textwidth]{./images/QVBoxLayout.png}

Κάθε στοιχείο τύπου QWidget προστίθεται πάνω στο layout με την κλήση της μεθόδου
\textbf{addWidget} και αντίστοιχα η αφαίρεση του γίνεται με την μέθοδο \textbf{removeWidget}.

\subsection{Δημιουργία λίστας αντικειμένων}
Μια λίστα QWidget μπορεί να αποθηκεύσει στην μνήμη στοιχεία τύπου QListWidgetItem.
Τα στοιχεία αυτά είναι αντικείμενα του Qt τα οποία δεν μπορούν να αποθηκευθούν σε
κάποια άλλη 


\begin{lstlisting}[language=C++, style=cppstyle]
#include <QListWidget>
#include <QListWidgetItem>


class MyWidget : public QWidget 
{
public:
    MyWidget(QWidget *parent = nullptr) : QWidget(parent) 
    {
        setFixedSize(400, 300);
        setWindowTitle("P2019140 - Konstantinos Tourtsakis");

        QListWidget *list_widget = new QListWidget(this);
        list_widget->addItem(new QListWidgetItem("Item 1"));
        list_widget->addItem(new QListWidgetItem("Item 2"));
        list_widget->addItem(new QListWidgetItem("Item 3"));
        list_widget->addItem(new QListWidgetItem("Item 4"));
        list_widget->addItem(new QListWidgetItem("Item 5"));
        list_widget->setGeometry(10, 10, 200, 200);
    }
};
\end{lstlisting}

\includegraphics[width=1.0\textwidth]{./images/QListWidget.png}

\subsection{Διάβασμα αρχείων από directory}
Το διάβασμα αρχείων από ένα directory γίνεται μέσω της QDir κλάσης στην οποία
αρχικοποιείται ένα αντικείμενο με το path του directory του οποίου θέλουμε να
διαβάσουμε. Στην συνέχεια αποθηκεύουμε τα αρχεία του directory σε μια λίστα
από QStrings και τα προσπελαύνουμε για την προσθήκη τους σε ένα QListWidget
με στόχο την προβολή τους στον χρήστη.
\begin{lstlisting}[language=C++, style=cppstyle]
#include <QDir>
#include <QStringList>


class MyWidget : public QWidget 
{
public:
    MyWidget(QWidget* parent = nullptr) : QWidget(parent) 
    {
        setFixedSize(300, 200);
        setWindowTitle("P2019140 - Konstantinos Tourtsakis");

        QVBoxLayout* layout = new QVBoxLayout(this);
        QListWidget* list_widget = new QListWidget(this);
        layout->addWidget(listWidget);

        QDir directory("C:\\Users\\kosta\\Documents\\Git\\Thesis\\source\\x64\\Debug");

        QStringList files = directory.entryList(QDir::Files);
        for (const QString& file : files) 
        {
            list_widget->addItem(file);
        }
    }
};
\end{lstlisting}

\includegraphics[width=1.0\textwidth]{./images/QDir_file_reading.png}

\subsection{Δημιουργία input field}
Βασικό στοιχείο κάθε εφαρμογής γραφικής διεπαφής αποτελεί ένα input field μιας
και επιτρέπει στον χρήστη να πληκτρολογήσει δεδομένα εισόδου για την εκτέλεση
μιας λειτουργίας του προγράμματος. Για τον σκοπό αυτό το Qt παρέχει τα αντικείμενα
τύπου QLineEdit. Όπως και τα QPushButton, τα αντικείμενα αυτά αρχικοποιούνται
και στην συνέχεια προστίθενται πάνω σε ένα πλαίσιο μέσω της μεθόδου addWidget.


\begin{lstlisting}[language=C++, style=cppstyle]
#include <QLineEdit>

class MyWidget : public QWidget 
{
public:
    MyWidget(QWidget* parent = nullptr) : QWidget(parent) 
    {
        setFixedSize(400, 300);
        setWindowTitle("P2019140 - Konstantinos Tourtsakis");

        QLineEdit* lineEdit = new QLineEdit(this);
        lineEdit->setPlaceholderText("Your input goes here");
        
        QVBoxLayout* layout = new QVBoxLayout(this);
        layout->addWidget(lineEdit);

        setLayout(layout);
    }
};
\end{lstlisting}
\includegraphics[width=1.0\textwidth]{./images/QLineEdit.png}


\section{Αποθήκευση στοιχείων στον δίσκο}
Κάθε εφαρμογή στις ημέρες μας χρειάζεται να αποθηκεύει την κατάσταση στην οποία
βρισκόταν την τελευταία φορά που εκτελέστηκε με στόχο να γίνει πιο κατανοητή η
χρήση του. Για τον σκοπό αυτό, το Qt φροντίζει να παρέχει τα απαραίτητα εργαλεία
για την αποθήκευση των στοιχείων της εφαρμογής βελτιώνοντας με αυτόν τον τρόπο
τον πηγαίο κώδικα που γράφεται για την προσθήκη της λειτουργίας αυτής. Παρακάτω
βλέπουμε μια μέθοδο που αποθηκεύει τα στοιχεία ενός Qt6 προγράμματος.


\begin{lstlisting}[language=C++, style=cppstyle]
void SaveSettings()
{
    QSettings settings("Company_name", "Application_title");
    settings.beginGroup("group_name");

    settings.setValue("WindowSize", window()->size());
    settings.setValue("Username", "JohnDoe");

    settings.endGroup("group_name");
}

\end{lstlisting}
Αρχικά δημιουργείται ένα αντικείμενο τύπου \textbf{QSettings}. Ακολουθεί ο ορισμός
της ομάδας στην οποία θα γίνουν οι αλλαγές και στην συνέχεια ορίζονται
οι τιμές πάνω στις οποίες θα γραφούν οι μεταβλητές του προγράμματος. Στο τέλος
επαναρχικοποιείται η ομάδα καλόντας την μέθοδο \textbf{endGroup}.

\section{Ανάκτηση εικονιδίων συστήματος αρχείων}
Όπως είναι γνωστό τα γραφικά περιβάλλοντα στηρίζονται πολύ στην χρήση εικονιδίων
για την διευκόλυνση του χρήστη κατά την περιήγησή του μιας και είναι πολύ πιο
αποτελεσματική η συσχέτιση μιας λειτουργίας με μια εικόνα παρά με κάποιο κείμενο.
Επομένως, είναι σημαντικό να υπάρχει μια συνάρτηση που επιστρέφει το εικονίδιο
ενός αρχείου δυναμικά, ώστε ο χρήστης να καταλαβαίνει από την αρχή την λειτουργία που
κρύβεται πίσω από την ενεργοποίηση του εικονιδίου αυτού. Παρακάτω βλέπουμε την συνάρτηση αυτή.


\begin{lstlisting}[language=C++, style=cppstyle]
QIcon GetFileIcon(const QString& file_path)
{
    SHFILEINFO shfi;
    memset(&shfi, 0, sizeof(SHFILEINFO));

    if (SHGetFileInfo(reinterpret_cast<const wchar_t*>(file_path.utf16()), 0, &shfi, sizeof(SHFILEINFO), SHGFI_ICON | SHGFI_USEFILEATTRIBUTES))
    {
        QPixmap pixmap = QPixmap::fromImage(QImage::fromHICON(shfi.hIcon)).scaled(QSize(PercentToWidth(6.66), PercentToHeight(11.85)), Qt::KeepAspectRatio, Qt::SmoothTransformation);
        QIcon icon(pixmap);

        // Cleanup the icon resource
        DestroyIcon(shfi.hIcon);

        return icon;
    }

    
    return QIcon();
}
\end{lstlisting}

Επειδή η εφαρμογή στην οποία αναφέρεται το παρόν κείμενο προορίζεται για χρήση σε
συστήματα Windows, η συγκεκριμένη συνάρτηση στηρίζεται πάνω σε αυτά. Αρχικά
δημιουργείται ένα struct τύπου SHFILEINFO πάνω στο οποίο θα αποθηκευτούν τα
δεδομένα του αρχείου, συμπεριλαμβανομένου και του εικονιδίου. Στην συνέχεια γίνεται
έλεγχος για την ανάκτηση του εικονιδίου από το σύστημα. Αφού η ανάκτηση γίνει με
επιτυχία, δημιουργείται ένα \textbf{QPixmap} με το εικονίδιο του αρχειόυ. Στην 
συνέχεια το pixmap χρησιμοποιείται για την δημιουργία του QIcon εικονιδίου το
οποίο και θα επιστραφεί από την συνάρτηση. Αν το εικονίδιο δεν βρεθεί η συνάρτηση
θα επιστρέψει ένα προεπιλεγμένο QIcon.


\section{Προτάσεις δεδομένων εισόδου σε input field}

Συχνά οι χρήστες βρίσκουν χρήσιμη την υποβοήθησή τους από την γραφική διεπαφή
στην περιήγησή τους στο εκάστοτε πρόγραμμα. Μια τέτοια βοήθεια μπορεί να συμβεί
επίσης στην είσοδο δεδομένων σε ένα input field. Ανάλογα με τις ενέργειες του
χρήστη υπάρχει κάποιος αλγόριθμος που καθορίζει την λίστα από την οποία το
input field θα προτείνει στον χρήστη δεδομένα εισόδου. Στο Qt αυτή η λειτουργία
υλοποιείται με ένα αντικείμενο QCompleter που ουσιαστικά συμπληρώνει αυτό που έχει
γράψει ο χρήστης με προτάσεις δεδομένων.

\begin{lstlisting}[language=C++, style=cppstyle]
    QLineEdit input_field;
    QStringList suggestions;
    suggestions << "Suggestions" << "for" << "thesis";
    
    QCompleter* completer = new QCompleter(suggestions, &input_field);
    completer->setCaseSensitivity(Qt::CaseInsensitive);
    
    input_field.setCompleter(completer);
\end{lstlisting}



\section{Επαναληπτική εκτέλεση διεργασιών παρασκηνίου}

Ένα βασικό πρόβλημα με τις βιβλιοθήκες εφαρμογών γραφικής διεπαφής είναι το γεγονός
πως αμέσως μετά την δημιουργία των γραφικών στοιχείων η βιβλιοθήκη αποκτά τον πλήρη
έλεγχο της εφαρμογής με αποτέλεσμα να μην είναι εφικτή η εκτέλεση κώδικα επαναληπτικά.
Αυτό καθιστά αδύνατη την εκτέλεση διεργασιών ρουτίνας οι οποίες πρέπει να εκτελούνται
σε κάθε επανάληψη του προγράμματος ώστε να γίνει εφικτή η υλοποίηση ορισμένων λειτουργιών.
Ευτυχώς οι περισσότερες βιβλιοθήκες, όπως και το Qt, διαθέτουν αντικείμενα που επιτρέπουν
την προσθήκη κώδικα που εκτελείται σε τακτά χρονικά διαστήματα που ορίζονται από τον
προγραμματιστή. Αυτό όπως είναι κατανοητό μπορεί να αξιοποιηθεί ορίζοντας ένα μηδενικό
χρονικό διάστημα, δημιουργόντας κατά κάποιο τρόπο μια επανάληψη που την διαχειρίζεται το
Qt.

 \begin{lstlisting}[language=C++, style=cppstyle]
QTimer* timer = new QTimer(this);
connect(timer, &QTimer::timeout, this, &MyWidget::TaskMainUserLoop);
timer->start(0);
\end{lstlisting}


\begin{lstlisting}[language=C++, style=cppstyle]
\end{lstlisting}



\chapter{Η βιβλιοθήκη XInput} \label{chapter:xinput}




\section{Τι είναι το XInput;}
Το XInput είναι μια βιβλιοθήκη διαχείρισης εντολών εισόδου για
συσκευές χειρισμού βιντεοπαιχνιδιών κατασκευασμένες από την Microsoft.
Οι συγκεκριμένες συσκευές είναι χειριστήρια Xbox οποιασδήποτε γεννιάς.

\section{Η κλάση του χειριστηρίου}



\chapter{Σχήματα και Πίνακες} \label{chapter:sximata}
\input{sximata}


\chapter{Τελευταία Μέρη} \label{chapter:telos}
\drop{Έ}
χοντας πλέον μελετήσει το πρόγραμμα αυτό και αποκτήσει μια καλή κατανόηση του
προβλήματος που προσπαθεί να λύσει, μπορούμε πια να δούμε καλύτερα πως θα μπορεί
να είναι χρήσιμο στην δική μας χρήση του Η/Υ. Η υλοποίηση αυτού του προγράμματος
αποτελεί όχι απλώς ένα παράδειγμα μιας ακόμη εφαρμογής εκτέλεσης προγραμμάτων αλλά
παρουσιάζει τα μειονεκτήματα των σύγχρονων γραφικών περιβαλλόντων που χρησιμοποιούν
τα δημοφιλή για οικιακή χρήση λειτουργικά συστήματα. Η εκτέλεση ενός προγράμματος
και η περιήγηση στο σύστημα αλλά και το ίδιο το πρόγραμμα είναι ένα σημαντικός παράγοντας
που καθορίζει την επιλογή του χρήστη ενός προσωπικού υπολογιστή. Οι εναλλακτικές που
υπάρχουν για την πρόσβαση σε λογισμικό με τις επιπρόσθετες περιφερειακές συσκευές
θέτουν τον χρήστη σε ένα περιβάλλον που περιορίζει τις δυνατότητές του λόγω της έλλειψης
σημαντικών desktop εφαρμογών ή της υπολογιστικής ισχύς που απαιτείται για την εκτέλεσή τους.
Όσον αφορά τις τεχνολογίες που χρησιμοποιήθηκαν για την υλοποίηση του προγράμματος αυτού:
\begin{itemize}
	\item
Το Qt είναι ένα εξαιρετικό εργαλείο για την δημιουργία εφαρμογών γραφικής διεπαφής. Προσφέρει
μια μεγάλη ποικιλία γραφικών στοιχείων που μπορούν να τροποποιηθούν για τις ανάγκες της εκάστοτε
εφαρμογής. Είναι αρκετά απλό και είναι επαρκές για την υλοποίηση πρακτικά κάθε λειτουργίας της
εφαρμογής. Επιπλέον, ως μια cross-platform βιβλιοθήκη αποτελεί μια καλή λύση που έχει την ευελιξία
να λειτουργήσει σε πολλαπλά συστήματα που υποστηρίζει.
	\item
Το XInput προσφέρει μια αρκετά κατανοητή λύση στην διαχείριση εισόδου από ένα χειριστήριο βιντεοπαιχνιδιών
τύπου Xbox και έπαιξε σημαντικό ρόλο στην επίτευξη του στόχου της εφαρμογής αυτής.
	\item
Το Windows είναι το λειτουργικό σύστημα που φιλοξένησε την εφαρμογή. Αρκετές λειτουργίες της βασίζονται πάνω
σε αυτό και ως το δημοφιλέστερο λειτουργικό σύστημα προσωπικού υπολογιστή αποτελεί ένα καλό παράδειγμα πάνω
στο οποίο παρουσιάστηκε η λύση του προβλήματος αυτού.
\end{itemize} 


\drop{H}
κύρια δομή της Πτυχιακής Εργασίας δεν μπορεί 
να προκαθοριστεί και ούτε θα ήταν σωστό 
ώστε να υπόκειται πάντα στη δημιουργικότητα
του φοιτητή υπό τη συμβουλή του επιβλέποντα.
Είναι όμως σημαντικό να υπάρχει ένα κεφάλαιο συμπερασμάτων στο 
οποίο να γίνεται τελικά σύνοψη της εργασίας και των συμπερασμάτων αυτής.
Βέβαια, μια τέτοια σύνοψη έλαβε χώρα και στην Εισαγωγή. 
Όμως τότε ο αναγνώστης δεν είχε διαβάσει την εργασία και ο
κύριος σκοπός ήταν να του προκαλέσει το ενδιαφέρον αλλά και να 
τον εισάγει ομαλά. 
Εδώ, όμως, ο σκοπός είναι να τον βοηθήσει όλα όσα διάβασε να τα καταχωρήσει
στο μυαλό του ακόμα καλύτερα.



Μετά τα συμπεράσματα, και το τέλος ουσιαστικά της Πτυχιακής Εργασίας,
υπάρχει ένας αριθμός από κεφάλαια τα οποία είναι χρήσιμα για τον αναγνώστη
και ουσιαστικά αποτελούν το ''κερασάκι στην τούρτα.''

\begin{itemize}
\item
Καταρχάς μπορεί να είναι κάποια επιπλέον κεφάλαια Παραρτήματος τα
οποία όμως απαριθμούνται αυτόνομα (δεν έχουν συσχέτιση με την
αρίθμηση των κεφαλαίων του κυρίου μέρους).
\item
Είναι οπωσδήποτε το κεφάλαιο της βιβλιογραφίας το οποίο αριθμεί της 
βιβλιογραφικές αναφορές κατά αύξοντα αριθμό πρώτης εμφάνισής τους στο κείμενο.
Αν π.χ. στο σημείο αυτό γινόταν η πρώτη αναφορά τότε εν μέσω του κειμένου
θα παρεμβάλλονταν το \cite{example}.
Η βιβλιογραφία μπορεί να είναι είτε στα αγγλικά είτε στα ελληνικά
ανάλογα με το προς αναφορά κείμενο.
\item
Το κεφάλαιο των συντμήσεων είναι ιδιαίτερα σημαντικό καθώς η Επιστήμη
της Πληροφορικής έχει πολλούς ξενικούς όρους που χρησιμοποιούμε καθημερινά με συντετμημένη
μορφή. Για παράδειγμα, η \textit{υπηρεσία ονομάτων περιοχής} (\tl{Domain Name Services -- DNS}).
Σε τέτοιες περιπτώσεις δίνουμε την ελληνική μετάφραση με πλάγια γράμματα και
στην παρένθεση έχουμε την αγγλική έκφραση και τη σύντμηση. Στη συνέχεια
είμαστε ελεύθεροι να χρησιμοποιήσουμε την ελληνική έκφραση ή τον συντμημένο 
αγγλικό τύπο. Προσοχή, όμως, γιατί χρησιμοποιούμε μόνο ένα από τα δύο! 
Το ίδιο κάνουμε και με τα ελληνικά, βάζοντας σε παρένθεση όμως μόνο τη
σύντμηση. Π.χ., Οργανισμός Τηλεπικοινωνιών Ελλάδος (ΟΤΕ). 
Στο κεφάλαιο των συντμήσεων έχουμε διαφορετικά μέρη για τις ελληνικές
και τις ξενικές συντμήσεις αλλά με αλφαβητική διάταξη.
\item
Το γλωσσάρι είναι ουσιαστικά η μετάφραση των ξενικών όρων
που συναντάμε στο κείμενο (είτε τους χρησιμοποιούμε γιατί
είθισται είτε όχι). Το γλωσσάρι δεν είναι μόνο λεξικό αλλά
είναι θεμιτό να περιέχει και μια μικρή ερμηνεία του όρου.
\item
Τέλος, είναι σημαντικό το ευρετήριο. \index{σημαντικό} \index{ευρετήριο} \index{ευρετήριο!σημαντικό}
Στο κεφάλαιο αυτό δίνονται με αλφαβητική διάταξη οι σελίδες που απαντώνται οι
σημαντικότεροι όροι μέσα στο κείμενο. Όχι κατ' ανάγκη όλοι οι σημαντικοί
όροι αλλά κυρίως εκείνοι που έχουν να κάνουν με την ουσία της Πτυχιακής Εργασίας.
\end{itemize} 



 




%CHAPTER
\chapter*{Παράρτημα Α'} \pagestyle{empty}
\addcontentsline{toc}{chapter}{Παράρτημα Α'}
Ενδεικτικό παράδειγμα παραρτήματος.
Έπεται το κεφάλαιο της βιβλιογραφίας.



%Bibliography
\begin{thebibliography}{99} \addcontentsline{toc}{chapter}{Βιβλιογραφία}
\pagestyle{headings}

%\bibitem{example}
%\tl{Authors,
%\textit{title}, information about the book, paper journal.}




\bibitem{example}
\tl{The Qt Company Ltd,
\textit{QApplication class}, Qt6 Documentation, https://doc.qt.io/qt-6/qapplication.html}



\bibitem{example}
\tl{The Qt Company Ltd,
\textit{QWidget class}, Qt6 Documentation, https://doc.qt.io/qt-6/qwidget.html}




\bibitem{example}
\tl{The Qt Company Ltd,
\textit{QPushButton class}, Qt6 Documentation, https://doc.qt.io/qt-6/qpushbutton.html}



\bibitem{example}
\tl{The Qt Company Ltd,
\textit{QVBoxLayout class}, Qt6 Documentation, https://doc.qt.io/qt-6/qvboxlayout.html}



\bibitem{example}
\tl{The Qt Company Ltd,
\textit{QHBoxLayout class}, Qt6 Documentation, https://doc.qt.io/qt-6/qhboxlayout.html}



\bibitem{example}
\tl{The Qt Company Ltd,
\textit{QGridLayout class}, Qt6 Documentation, https://doc.qt.io/qt-6/qgridlayout.html}



\bibitem{example}
\tl{The Qt Company Ltd,
\textit{QListWidget class}, Qt6 Documentation, https://doc.qt.io/qt-6/qlistwidget.html}


\bibitem{example}
\tl{The Qt Company Ltd,
\textit{QDir class}, Qt6 Documentation, https://doc.qt.io/qt-6/qdir.html}



\bibitem{example}
\tl{The Qt Company Ltd,
\textit{QLineEdit class}, Qt6 Documentation, https://doc.qt.io/qt-6/qlineedit.html}



\bibitem{example}
\tl{The Qt Company Ltd,
\textit{QSettings class}, Qt6 Documentation, https://doc.qt.io/qt-6/qsettings.html}



\bibitem{example}
\tl{The Qt Company Ltd,
\textit{QIcon class}, Qt6 Documentation, https://doc.qt.io/qt-6/qicon.html}



\bibitem{example}
\tl{Ivan Akulov,
\textit{Extracting icon using WinAPI in Qt app}, Stack Overflow ερώτηση, https://stackoverflow.com/questions/12459145/extracting-icon-using-winapi-in-qt-app}



\bibitem{example}
\tl{The Qt Company Ltd,
\textit{QCompleter Class}, Qt6 Documentation, https://doc.qt.io/qt-6/qcompleter.html}



\bibitem{example}
\tl{The Qt Company Ltd,
\textit{QTimer Class}, Qt6 Documentation, https://doc.qt.io/qt-6/qtimer.html}





\bibitem{example}
\tl{Microsoft,
\textit{Getting Started With XInput in Windows applications}, Windows Win32 Documentation, https://learn.microsoft.com/en-us/windows/win32/xinput/getting-started-with-xinput}




\bibitem{example}
\tl{Minalien,
\textit{Xbox 360 Controller Input in C++ with XInput}, Code Project article, https://www.codeproject.com/articles/26949/xbox-360-controller-input-in-c-with-xinput}




\bibitem{example}
\tl{Microsoft,
\textit{SendInput function (winuser.h)}, Windows Win32 Documentation, https://learn.microsoft.com/en-us/windows/win32/api/winuser/nf-winuser-sendinput}


\end{thebibliography}


%CHAPTER
\chapter*{Συντμήσεις} \pagestyle{empty}
\addcontentsline{toc}{chapter}{Συντμήσεις}




\gl
{\tl{DNS}}
{\tl{Domain Name Server}}


\gl
{\tl{UI}}
{\tl{User Interface}}


%CHAPTER
\chapter*{Γλωσσάρι Ξενικών Όρων} \pagestyle{empty}
\addcontentsline{toc}{chapter}{Γλωσσάρι Ξενικών Όρων}
%\Huge{\tl{A}}
%\normalsize

\gl
{\tl{Access Permissions}}
{Δικαιώματα Πρόσβασης}



\gl
{\tl{Αrguments}}
{Παράμετροι}




\gl
{\tl{Flags}}
{}





\newpage
\addcontentsline{toc}{chapter}{Ευρετήριο}
\printindex



\end{document}
