\drop{Τ}{Α} 
σχήματα είναι απαραίτητα πολλές φορές για να κατανοήσει
ο αναγνώστης καλύτερα το περιεχόμενο του κειμένου μας.
Μερικές φορές είναι απαραίτητα και για το κείμενό μας.
Επ ουδενί όμως, δεν επιτρέπεται η αντιγραφή (\tl{scan})
σχημάτων άλλων συγγραφέων που υπάρχουν σε βιβλία,
επιστημονικές εργασίες κτλ. 

Όπως τα σχήματα, χρήσιμοι είναι και οι πίνακες 
γιατί μπορεί κάποιος με εύκολο τρόπο να βρει χρήσιμη πληροφορίας,
όπως π.χ. παράθεση πειραματικών αποτελεσμάτων που από τη 
φύση της είναι δύσκολη και ίσως δυσνόητη όταν 
βρίσκεται εντός του κειμένου.

\section{Σχήματα}

Στη συνέχεια θα δωθούν μερικά παραδείγμα μορφής σχημάτων.

Το ``κλασικό'' σχήμα είναι αυτό που
απεικονίζεται στο Σχήμα \ref{fig:intro_data_flow}.
Προφανώς, το σχήμα μπορεί να θέλει να είναι λίγο μικρότερο
(π.χ., όπως στο Σχήμα \ref{fig:intro_data_flow2})
αλλά φροντίζουμε να μην είναι το πλάτος του μικρότερο από
το 1/2 του πλάτους του κειμένου. Σε κάθε περίπτωση, 
τα σχήματα αριθμούνται ξεχωριστά για κάθε κεφάλαιο
αρχίζοντας από το 1 με πρόθεμα το νούμερο του κεφαλαίου.

\myfig{intro_data_flow}
{Ένα δίκτυο.}

\myfigadapt{intro_data_flow2}
{To ίδιο δίκτυο  που
απεικονίζεται στο Σχήμα \ref{fig:intro_data_flow}
αλλά λίγο μικρότερο.
}
{0.5}




Υπάρχουν περιπτώσεις που μπορεί κάποιος να
θέλει δύο ίδια σχήματα δίπλα-δίπλα,
όπως είναι η περίπτωση του Σχήματος \ref{fig:intro_data_flow3}.
Προφανώς μπορεί να γίνει ξεχωριστή αναφορά στα
δύο υπο-σχήματα \ref{fig:intro_data_flow}.α και \ref{fig:intro_data_flow}.β
με τον αυτό τρόπο.
Το κείμενο για τα δύο υποσχήματα είναι προαιρετικό αλλά η αρίθμηση υποχρεωτική.

\mytfig{intro_data_flow3}
{Το ίδιο δίκτυο σε διπλή απεικόνιση.}
{Κείμενο για το πρώτο.}
{Κείμενο για το δεύτερο.}

Όπως προηγουμένως, μπορεί να είναι επιθυμητή
η αλλαγή του μεγέθους των σχημάτων, όπως φαίνεται στο 
Σχήμα \ref{fig:intro_data_flow4}.


\mytfigadapt{intro_data_flow4}
{Το ίδιο δίκτυο σε διπλή απεικόνιση σε σμίκρυνση
και δίχως κείμενο για τα υπο-σχήματα.}
{}
{}
{0.3}
{0.3}


Τέλος, υπάρχει το ενδεχόμενο να είναι αναγκαία 
η παρουσίαση κάποιων σχημάτων σε τετράδα,
όπως είναι το Σχήμα \ref{fig:intro_data_flow5}.

\myffig{intro_data_flow5}
{Το ίδιο δίκτυο σε τετραπλή απεικόνιση.}
{Κείμενο για το πρώτο.}
{Κείμενο για το δεύτερο.}
{Κείμενο για το τρίτο.}
{Κείμενο για το τέταρτο.}

Με όμοιο τρόπο όπως προηγουμένως μπορεί να επιτευχθεί
αλλαγή του μεγέθους.


\section{Πίνακες}

Οι πίνακες ομοιάζουν με το κλασικό σχήμα στη μορφή και
την αναφορά σε αυτούς, με μόνη διαφορά πως προηγούνται αντί να έπονται,
όπως φαίνεται και στο παράδειγμα του Πίνακα \ref{tbl:example}.

%table
\begin{table}
\caption{Παράδειγμα Πίνακα.}
\label{tbl:example}
\begin{center}
\begin{tabular}{|p{18mm}|p{12mm}|p{12mm}|p{12mm}|p{12mm}|p{12mm}|p{12mm}|p{12mm}|}\hline
		& 	$s_0$&	$s_1$& $s_2$	& $s_3$	& $s_4$	& $s_5$	& $s_6$	\\ \cline{1-8}
$f_0(s_\chi)$	& 	$1$	&	$2$	& $3$	& $4$	& $5$	& $6$	& $0$	\\ \cline{1-8}
$f_1(s_\chi)$	&	$1$	&	$3$	& $5$	& $0$	& $2$	& $4$	& $6$	\\ \cline{1-8}
$f_2(s_\chi)$	&	$4$	&	$0$	& $3$	& $6$	& $2$	& $5$	& $1$	\\ \cline{1-8}
$f_3(s_\chi)$	&	$3$	&	$0$	& $4$	& $1$	& $5$	& $2$	& $6$	\\ \cline{1-8}
$f_4(s_\chi)$	&	$2$	&	$5$	& $1$	& $4$	& $0$	& $3$	& $6$	\\ \cline{1-8}
$f_5(s_\chi)$	&	$2$	&	$3$	& $4$	& $5$	& $6$	& $0$	& $1$	\\ \cline{1-8}
$f_6(s_\chi)$	&	$6$	&	$4$	& $2$	& $0$	& $5$	& $3$	& $1$	\\ \cline{1-8}
$f_7(s_\chi)$	&	$1$	&	$0$	& $6$	& $5$	& $4$	& $3$	& $2$	\\ \cline{1-8}
$f_8(s_\chi)$	&	$5$	&	$6$	& $0$	& $1$	& $2$	& $3$	& $4$	\\ \cline{1-8}
$f_9(s_\chi)$	&	$3$	&	$2$	& $1$	& $0$	& $6$	& $5$	& $4$	\\ \hline
\end{tabular}
\end{center}
\end{table}
