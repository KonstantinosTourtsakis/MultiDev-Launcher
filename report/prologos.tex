Η προσπάθεια δημιουργίας ενός προτύπου οδηγιών για τη συγγραφή μιας Πτυχιακής Εργασίας,
τόσο ως προς τη μορφή όσο και ως προς
το περιεχόμενο και τη διάταξη αυτού, ξεκίνησε ιδιαίτερα νωρίς στο 
Τμήμα Πληροφορικής του Ιονίου Πανεπιστημίου.
Ο κύριος σκοπός 
είναι να τεθεί 
με σαφήνεια αλλά και από νωρίς στους φοιτητές τους Τμήματος 
οι δυσκολίες που πρέπει να περάσουν προκειμένου να
τελεσφορήσει επιτυχώς η πτυχιακή τους δοκιμασία 
αλλά και να μη μειωθεί στο ελάχιστο το όποιο αποτέλεσμα
των προσπαθειών τους λόγω απειρίας στη συγγραφή 
επιστημονικών κειμένων υψηλού επιπέδουν όπως είναι
η Πτυχιακή Εργασία.

Η εκπόνηση της Πτυχιακής Εργασίας είναι ουσιαστικά μια δοκιμασία 
όχι μόνο λαμβάνοντας υπόψη τις δυσκολίες 
που συναντά ο φοιτητής αλλά το επίπεδο του αποτελέσματος στο οποίο
θα πρέπει να φτάσει ώστε να θεωρηθεί επιτυχής η προσπάθειά του αυτή.
Η Πτυχιακή Εργασία δεν είναι όπως μια οποιαδήποτε άλλη εργασία, 
αλλά αφού εκπονηθεί θα παραδωθεί στη Βιβλιοθήκη του Τμήματος και θα 
αποτελεί κτήμα γνώσης για τις επόμενες γενιές. Θα είναι
ένα δημόσιο έγγραφο ουσιαστικά στο οποίο όμως θα μπορεί να
έχει πρόσβαση ο καθένας. Εκ τούτου εύλογα προκύπτει πως όχι μόνο
το περιεχόμενο αλλά και η ίδια η μορφή της θα πρέπει να είναι
προσεγμένη, μάλλον αυστηρή, περίκαλλος και σίγουρα να προσάδει
επιστημονοσύνης.

Ο φοιτητής ερχόμενος προς τη δοκιμασία αυτή κατά το τελευταίο έτος
των σπουδών του θα πρέπει να γνωρίζει πως στο πρόσωπο 
του επιβλέποντος καθηγητή του θα βρει έναν
αξιόλογο συμπαραστάτη και συμβουλάτορα.
Η ίδια η διαδικασία απαιτεί συμβουλές και ένα προχωρημένο επίπεδο,
αρκεί να αναλογιστεί
κανείς πως ο βαθμός δεν δίδεται από έναν και μόνο διδάσκοντα αλλά θα
πρέπει να συναινέσει μία τριμελής επιτροπή στην οποία όλα τα
μέλη έχουν την ίδια βαρύτητα ψήφου.
Επίσης, η διαδικασία εξέτασης της Πτυχιακής Εργασίας είναι δημόσια
στη διάρκεια της οποίας ο φοιτητής θα πρέπει να απαντήσει 
πειστικά σε ερωτήσεις και σχόλια.

Τα παραπάνω δεν θα πρέπει να φοβίζουν τους φοιτητές αλλά μάλλον να αποτελούν
πρόκληση για την εκπόνηση πραγματικά αξιόλογων εργασιών.
Η εμπειρία έχει δείξει πως τελικά η εκπόνηση μιας καλής Πτυχιακής Εργασίας,
βοηθάει τους αποφοίτους στα πρώτα επαγγελματικά τους βήματα,
τα οποία είναι ιδιαίτερα σημαντικά για την καριέρα τους.

Το παρόν πρότυπο απέχει μακράν της τελειότητας
αλλά είναι μια πρώτη προσπάθεια να τεθούν 
ορισμένοι κανόνες ώστε η επίπονη προσπάθεια των
φοιτητών κατά τη διάρκεια της Πτυχιακής Εργασίας
να μην απαξιωθεί στο ελάχιστο λόγω ελλιπούς προβολής.
Οποιαδήποτε σχόλια και προτάσεις είναι δεκτά στο \tl{email: okon@ionio.gr} ώστε να
καταφέρουμε οι διδάσκοντες να βελτιώνουμε συν τω χρόνω
το επίπεδο σπουδών του Τμήματός μας. 


Με τις σκέψεις αυτές και με πολλές ευχές
από όλους τους συναδέλφους 
διδάσκοντες του Τμήματος Πληροφορικής
 για καλή επιτυχία στο επίπονο έργο της
Πτυχιακής Εργασίας σας ευχόμαστε Καλό Πτυχίο!!!
\bigskip \\
Κέρκυρα, \today \\
Κωνσταντίνος Τουρτσάκης
 
