



\drop{Ε}{ίναι}
πλέον σίγουρο πως η χρήση ενός προσωπικού υπολογιστή γίνεται μέσω του
πληκτρολογίου σε συνδυασμό με την χρήση του ποντικιού για την περιήγηση του
χρήστη μέσα στο γραφικό περιβάλλον των σύγχρονων συστημάτων. Όμως αυτό ήταν
ανέκαθεν μια μέθοδος περιήγησης που προορίζονταν για εργασιακή χρήση. Με την
εξέλιξη των τεχνολογιών και την εισαγωγή ολοένα και περισσότερων πολυμεσικών
εφαρμογών στο περιβάλλον του Η/Υ, ήταν αναπόφευκτη η μετάβαση σε μια εποχή
όπου ο Η/Υ έχει εφαρμογή σε κάθε σπίτι ανεξαρτήτως του επαγγέλματος του
ιδιωκτήτη. Αναγνωρίζοντας την αλλαγή αυτή και το γεγονός πως παραμένουν
περιθώρια βελτίωσης από την πλευρά του συστήματος ώς προς την επικοινωνία
μεταξύ του χρήστη και του υπολογιστή, το πρόγραμμα αυτό έχει ως στόχο την
εκτέλεση εφαρμογών ανεξαρτήτος της συσκευής εισόδου του χρήστη. Επομένως,
απώτερος σκοπός είναι η αξιοποίηση των νέων περιφειακών συσκευών στην
εκτέλεση και περιήγηση του λειτουργικού συστήματος αλλά και την υλοποίηση
νέων διαδικασιών εκτέλεσης εφαρμογών από προϋπάρχουσες συσκευές, όπως το
πληκτρολόγιο και το ποντίκι. Για την υλοποίηση αυτής της εφαρμογής είναι
απαραίτητο να γίνει χρήση τεχνολογιών για την γραφική διεπαφή χρήστη και την
είσοδο δεδομένων από τις περιφερειακές συσκευές στο σύστημα. Για την
εξυπηρέτηση αυτών των αναγκών γίνεται κυρίως χρήση της βιβλιοθήκης Qt6, του
XInput και της βιβλιοθήκης των Windows.






