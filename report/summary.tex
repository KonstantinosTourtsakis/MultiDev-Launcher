







The use of personal computers has remained the same for decades. However, unlike the times when 
computers were devices that few had access to, they are now present in a large percentage of 
people's homes. Therefore, the intended use has diversified, making it necessary for the user 
to navigate the system according to the input devices that suit and justify this use. This 
program aims to solve this problem by adding functionality that allows the user to navigate 
using either the keyboard, the mouse, or a video game console controller. For the implementation 
of this program, the Qt graphical interface library, the XInput library for managing Xbox 
controller input commands, and the Windows system on which the application will focus will be 
utilized. Qt is a cross-platform library for developing graphical interface applications and 
provides everything needed for an application like this one. Upon entering the program the user is 
prompted to enter the username that the application will use. They then enter the main window 
of the application which is primarily navigated using the mouse and the controller. This window 
includes three tabs. One with the user's applications, one with the user's favorite applications and 
one with the application's settings where they can customize it. In settings there is also 
a button to open the virtual keyboard window, which is accessible via the video game controller too. 
Additionally, there is a window for executing applications via the keyboard. This window makes 
the execution of applications using this device easier by adding navigation functionality that addresses 
issues such as focusing on the wrong objects in the graphical interface. The application's code 
includes a class for creating the program's graphical interface, a class for creating the virtual 
keyboard, and a class for controlling input commands from the video game controller. Additionally, 
there are functions that "simulate" keystrokes which are used to map the controller button presses 
to shortcuts available in the system. Finally, this application can be installed via an installer 
created for this purpose. This installer is available in the corresponding repository on 
GitHub \footnote{https://github.com/KonstantinosTourtsakis/MultiDev-Launcher/} and includes the source 
code of the project.