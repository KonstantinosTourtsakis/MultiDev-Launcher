\drop{Έ}
χοντας πλέον μελετήσει το πρόγραμμα αυτό και αποκτήσει μια καλή κατανόηση του
προβλήματος που προσπαθεί να λύσει, μπορούμε πια να δούμε καλύτερα πώς μπορεί
να είναι χρήσιμο στη δική μας χρήση του Η/Υ. Η υλοποίηση αυτού του προγράμματος
αποτελεί όχι απλώς ένα παράδειγμα μιας ακόμη εφαρμογής εκτέλεσης προγραμμάτων αλλά
παρουσιάζει και τα μειονεκτήματα των σύγχρονων γραφικών περιβαλλόντων που χρησιμοποιούν
τα δημοφιλή για οικιακή χρήση λειτουργικά συστήματα. Η εκτέλεση ενός προγράμματος, η περιήγηση
σε αυτό αλλά και η περιήγηση στο λειτουργικό σύστημα είναι σημαντικοί παράγοντες
που καθορίζουν την επιλογή του χρήστη ενός προσωπικού υπολογιστή. Οι εναλλακτικές λύσεις που
υπάρχουν για την πρόσβαση σε λογισμικό με τις επιπρόσθετες περιφερειακές συσκευές
θέτουν τον χρήστη σε ένα περιβάλλον που περιορίζει τις δυνατότητές του λόγω της έλλειψης
σημαντικών desktop εφαρμογών ή της υπολογιστικής ισχύος που απαιτείται για την εκτέλεσή τους.
Παραδείγματα δεν αποτελούν μόνο οι εναλλακτικές στα λειτουργικά συστήματα αλλά είναι επίσης
όλες οι συσκευές που παρέχουν πρόσβαση σε δημοφιλείς εφαρμογές, συσκευές όπως ένα έξυπνο κινητό
τηλέφωνο, μια κονσόλα βιντεοπαιχνιδιών ή ένα TV Box.

Όσον αφορά τις τεχνολογίες που χρησιμοποιήθηκαν για την υλοποίηση του προγράμματος αυτού:
\begin{itemize}
	\item
Το Qt είναι ένα εξαιρετικό εργαλείο για τη δημιουργία εφαρμογών γραφικής διεπαφής. Προσφέρει
μια μεγάλη ποικιλία γραφικών στοιχείων που μπορούν να τροποποιηθούν για τις ανάγκες της εκάστοτε
εφαρμογής.
	\item
Το XInput προσφέρει μια αρκετά κατανοητή λύση στην διαχείριση εισόδου από ένα χειριστήριο βιντεοπαιχνιδιών
τύπου Xbox και έπαιξε σημαντικό ρόλο στην επίτευξη του στόχου της εφαρμογής αυτής.
	\item
Το Windows ως το δημοφιλέστερο λειτουργικό σύστημα προσωπικού υπολογιστή αποτελεί ένα καλό παράδειγμα πάνω
στο οποίο παρουσιάστηκε η λύση του προβλήματος αυτού.
\end{itemize} 

 
Στο μέλλον η εφαρμογή που υλοποιήθηκε μπορεί να επεκταθεί υποστηρίζοντας περισσότερες περιφερειακές
συσκευές. Συσκευές όπως χειριστήρια τύπου τηλεχειριστήριο προσθέτουν ένα ακόμη επίπεδο ευκολίας στην
περιήγηση αφού τα ίδια είναι σχεδιασμένα για την πιο απλή και κατανοητή χρήση από τον χρήστη. Επιπλέον,
με την φωνητική αναγνώριση να είναι αξιόπιστη στις μέρες μας μια τέτοιου είδους λειτουργικότητα βασικής
περιήγησης θα ήταν εξαιρετικά χρήσιμη για ανθρώπους που έχουν κάποιο σωματικό μειονέκτημα όπως ακρωτηριασμένα
χέρια, όπου στην περίπτωση αυτή δεν είναι ικανοί να χρησιμοποιήσουν οποιαδήποτε από τις περιφερειακές συσκευές
που είδαμε μέχρι τώρα. Αναπτύσσοντας περαιτέρω παρόμοιους τρόπους περιήγησης του συστήματος, αυτό συνεισφέρει στην
συμπερίληψη όσων περισσότερων ανθρώπων γίνεται στον κόσμο του λογισμικού. Τέλος, η γραφική διεπαφή μπορεί
να βελτιωθεί και οι ρυθμίσεις μπορούν να εμπλουτιστούν με περισσότερες επιλογές παραμετροποίησης της εφαρμογής
ώστε να υπάρχει μεγαλύτερη εξατομίκευση από πιο έμπειρους χρήστες.