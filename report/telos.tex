\drop{H}
κύρια δομή της Πτυχιακής Εργασίας δεν μπορεί 
να προκαθοριστεί και ούτε θα ήταν σωστό 
ώστε να υπόκειται πάντα στη δημιουργικότητα
του φοιτητή υπό τη συμβουλή του επιβλέποντα.
Είναι όμως σημαντικό να υπάρχει ένα κεφάλαιο συμπερασμάτων στο 
οποίο να γίνεται τελικά σύνοψη της εργασίας και των συμπερασμάτων αυτής.
Βέβαια, μια τέτοια σύνοψη έλαβε χώρα και στην Εισαγωγή. 
Όμως τότε ο αναγνώστης δεν είχε διαβάσει την εργασία και ο
κύριος σκοπός ήταν να του προκαλέσει το ενδιαφέρον αλλά και να 
τον εισάγει ομαλά. 
Εδώ, όμως, ο σκοπός είναι να τον βοηθήσει όλα όσα διάβασε να τα καταχωρήσει
στο μυαλό του ακόμα καλύτερα.



Μετά τα συμπεράσματα, και το τέλος ουσιαστικά της Πτυχιακής Εργασίας,
υπάρχει ένας αριθμός από κεφάλαια τα οποία είναι χρήσιμα για τον αναγνώστη
και ουσιαστικά αποτελούν το ''κερασάκι στην τούρτα.''

\begin{itemize}
\item
Καταρχάς μπορεί να είναι κάποια επιπλέον κεφάλαια Παραρτήματος τα
οποία όμως απαριθμούνται αυτόνομα (δεν έχουν συσχέτιση με την
αρίθμηση των κεφαλαίων του κυρίου μέρους).
\item
Είναι οπωσδήποτε το κεφάλαιο της βιβλιογραφίας το οποίο αριθμεί της 
βιβλιογραφικές αναφορές κατά αύξοντα αριθμό πρώτης εμφάνισής τους στο κείμενο.
Αν π.χ. στο σημείο αυτό γινόταν η πρώτη αναφορά τότε εν μέσω του κειμένου
θα παρεμβάλλονταν το \cite{example}.
Η βιβλιογραφία μπορεί να είναι είτε στα αγγλικά είτε στα ελληνικά
ανάλογα με το προς αναφορά κείμενο.
\item
Το κεφάλαιο των συντμήσεων είναι ιδιαίτερα σημαντικό καθώς η Επιστήμη
της Πληροφορικής έχει πολλούς ξενικούς όρους που χρησιμοποιούμε καθημερινά με συντετμημένη
μορφή. Για παράδειγμα, η \textit{υπηρεσία ονομάτων περιοχής} (\tl{Domain Name Services -- DNS}).
Σε τέτοιες περιπτώσεις δίνουμε την ελληνική μετάφραση με πλάγια γράμματα και
στην παρένθεση έχουμε την αγγλική έκφραση και τη σύντμηση. Στη συνέχεια
είμαστε ελεύθεροι να χρησιμοποιήσουμε την ελληνική έκφραση ή τον συντμημένο 
αγγλικό τύπο. Προσοχή, όμως, γιατί χρησιμοποιούμε μόνο ένα από τα δύο! 
Το ίδιο κάνουμε και με τα ελληνικά, βάζοντας σε παρένθεση όμως μόνο τη
σύντμηση. Π.χ., Οργανισμός Τηλεπικοινωνιών Ελλάδος (ΟΤΕ). 
Στο κεφάλαιο των συντμήσεων έχουμε διαφορετικά μέρη για τις ελληνικές
και τις ξενικές συντμήσεις αλλά με αλφαβητική διάταξη.
\item
Το γλωσσάρι είναι ουσιαστικά η μετάφραση των ξενικών όρων
που συναντάμε στο κείμενο (είτε τους χρησιμοποιούμε γιατί
είθισται είτε όχι). Το γλωσσάρι δεν είναι μόνο λεξικό αλλά
είναι θεμιτό να περιέχει και μια μικρή ερμηνεία του όρου.
\item
Τέλος, είναι σημαντικό το ευρετήριο. \index{σημαντικό} \index{ευρετήριο} \index{ευρετήριο!σημαντικό}
Στο κεφάλαιο αυτό δίνονται με αλφαβητική διάταξη οι σελίδες που απαντώνται οι
σημαντικότεροι όροι μέσα στο κείμενο. Όχι κατ' ανάγκη όλοι οι σημαντικοί
όροι αλλά κυρίως εκείνοι που έχουν να κάνουν με την ουσία της Πτυχιακής Εργασίας.
\end{itemize} 



 
