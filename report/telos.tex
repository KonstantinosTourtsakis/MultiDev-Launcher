\drop{Έ}
χοντας πλέον μελετήσει το πρόγραμμα αυτό και αποκτήσει μια καλή κατανόηση του
προβλήματος που προσπαθεί να λύσει, μπορούμε πια να δούμε καλύτερα πως θα μπορεί
να είναι χρήσιμο στην δική μας χρήση του Η/Υ. Η υλοποίηση αυτού του προγράμματος
αποτελεί όχι απλώς ένα παράδειγμα μιας ακόμη εφαρμογής εκτέλεσης προγραμμάτων αλλά
παρουσιάζει τα μειονεκτήματα των σύγχρονων γραφικών περιβαλλόντων που χρησιμοποιούν
τα δημοφιλή για οικιακή χρήση λειτουργικά συστήματα. Η εκτέλεση ενός προγράμματος
και η περιήγηση στο σύστημα αλλά και το ίδιο το πρόγραμμα είναι ένα σημαντικός παράγοντας
που καθορίζει την επιλογή του χρήστη ενός προσωπικού υπολογιστή. Οι εναλλακτικές που
υπάρχουν για την πρόσβαση σε λογισμικό με τις επιπρόσθετες περιφερειακές συσκευές
θέτουν τον χρήστη σε ένα περιβάλλον που περιορίζει τις δυνατότητές του λόγω της έλλειψης
σημαντικών desktop εφαρμογών ή της υπολογιστικής ισχύς που απαιτείται για την εκτέλεσή τους.
Όσον αφορά τις τεχνολογίες που χρησιμοποιήθηκαν για την υλοποίηση του προγράμματος αυτού:
\begin{itemize}
	\item
Το Qt είναι ένα εξαιρετικό εργαλείο για την δημιουργία εφαρμογών γραφικής διεπαφής. Προσφέρει
μια μεγάλη ποικιλία γραφικών στοιχείων που μπορούν να τροποποιηθούν για τις ανάγκες της εκάστοτε
εφαρμογής. Είναι αρκετά απλό και είναι επαρκές για την υλοποίηση πρακτικά κάθε λειτουργίας της
εφαρμογής. Επιπλέον, ως μια cross-platform βιβλιοθήκη αποτελεί μια καλή λύση που έχει την ευελιξία
να λειτουργήσει σε πολλαπλά συστήματα που υποστηρίζει.
	\item
Το XInput προσφέρει μια αρκετά κατανοητή λύση στην διαχείριση εισόδου από ένα χειριστήριο βιντεοπαιχνιδιών
τύπου Xbox και έπαιξε σημαντικό ρόλο στην επίτευξη του στόχου της εφαρμογής αυτής.
	\item
Το Windows είναι το λειτουργικό σύστημα που φιλοξένησε την εφαρμογή. Αρκετές λειτουργίες της βασίζονται πάνω
σε αυτό και ως το δημοφιλέστερο λειτουργικό σύστημα προσωπικού υπολογιστή αποτελεί ένα καλό παράδειγμα πάνω
στο οποίο παρουσιάστηκε η λύση του προβλήματος αυτού.
\end{itemize} 

 
