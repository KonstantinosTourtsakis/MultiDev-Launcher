






Η χρήση του προσωπικού υπολογιστή έχει παραμείνει η ίδια εδώ και δεκάδες χρόνια.
Σε αντίθεσή όμως με τις εποχές που οι υπολογιστές αποτελούσαν συσκευές που λίγοι
είχαν στη διάθεσή τους, πλέον υπάρχουν σε ένα μεγάλο ποσοστό στα σπίτια των
ανθρώπων. Η χρήση για την οποία προορίζεται επομένως έχει διαφοροποιηθεί και
για αυτόν τον λόγο είναι αναγκαίο να μπορεί ο χρήστης να περιηγηθεί μέσα στο σύστημα
ανάλογα με τις συσκευές εισόδου που ικανοποιούν και αιτιολογούν την χρήση αυτή.
Το πρόγραμμα που αναπτύχθηκε στο πλαίσιο της παρούσας πτυχιακής εργασίας 
προσπαθεί να λύσει αυτό το πρόβλημα, προσθέτοντας λειτουργικότητα
με την οποία ο χρήστης θα μπορεί να περιηγηθεί χρησιμοποιώντας είτε το πληκτρολόγιο,
είτε το ποντίκι είτε ένα χειριστήριο κονσόλας βιντεοπαιχνιδιών. Για την υλοποίηση του
προγράμματος αυτού θα αξιοποιηθεί η βιβλιοθήκη γραφικής διεπαφής Qt, η βιβλιοθήκη
XInput με το οποίο γίνεται διαχείριση εντολών εισόδου χειριστηρίων Xbox και το
σύστημα Windows στο οποίο θα εστιάσει η εφαρμογή. Το Qt είναι μια cross-platform βιβλιοθήκη
για ανάπτυξη εφαρμογών γραφικής διεπαφής και παρέχει οτιδήποτε χρειάζεται μια τέτοιου είδους εφαρμογή. Ο χρήστης κατά την εισαγωγή του
στο πρόγραμμα καλείται να εισάγει το όνομα χρήστη που θα χρησιμοποιήσει η εφαρμογή.
Στην συνέχεια εισέρχεται στο βασικό παράθυρο της εφαρμογής το οποίο αξιοποιείται κυρίως
από το ποντίκι και  το χειριστήριο. Το παράθυρο αυτό έχει 3 καρτέλες. Μία με τις εφαρμογές
του χρήστη, μία με τις αγαπημένες εφαρμογές του χρήστη και μία με τις ρυθμίσεις της εφαρμογής
στις οποίες μπορεί να την παραμετροποιήσει . Στις ρυθμίσεις μπορεί επίσης να βρει και ένα
κουμπί για το άνοιγμα του παραθύρου ενός εικονικού πληκτρολογίου στο οποίο έχει πρόσβαση και
μέσω του χειριστηρίου βιντεοπαιχνιδιών. Επιπλέον υπάρχει ένα παράθυρο για την εκτέλεση 
εφαρμογών μέσω του πληκτρολογίου. Αυτό το παράθυρο διευκολύνει την εκτέλεση των εφαρμογών μέσω της
συσκευής αυτής προσθέτοντας λειτουργικότητα περιήγησης και αντιμετωπίζοντας προβλήματα
όπως εστίαση σε λάθος αντικείμενα πάνω στην γραφική διεπαφή. Στον κώδικα της εφαρμογής υπάρχει
μια κλάση στην οποία δημιουργείται η γραφική διεπαφή του προγράμματος αυτού, μια κλάση για την
δημιουργία του εικονικού πληκτρολογίου και μια κλάση για τον έλεγχο εντολών εισόδου από το
χειριστήριο βιντεοπαιχνιδιών. Επιπλέον υπάρχουν συναρτήσεις που "προσομοιώνουν" πατήματα πλήκτρων
και χρησιμεύουν στην αντιστοίχηση των πιέσεων των κουμπιών του χειριστηρίου σε συντομεύσεις που
υπάρχουν στο σύστημα. Τέλος, η εφαρμογή αυτή μπορεί να εγκατασταθεί μέσω ενός installer που έχει
δημιουργηθεί για τον σκοπό αυτό. Ο installer αυτός είναι διαθέσιμος στο αντίστοιχο αποθετήριο που
υπάρχει στο GitHub \footnote{https://github.com/KonstantinosTourtsakis/MultiDev-Launcher/} και περιλαμβάνει τον πηγαίο κώδικα της εργασίας.






%\cleardoublepage
\newpage
\section{English}
The use of personal computers has remained the same for decades. However, unlike the times when computers were devices that few had access to, they are now present in a large percentage of people's homes. Therefore, the intended use has diversified, making it necessary for the user to navigate the system according to the input devices that suit and justify this use.

This program aims to solve this problem by adding functionality that allows the user to navigate using either the keyboard, the mouse, or a video game console controller. For the implementation of this program, the Qt graphical interface library, the XInput library for managing Xbox controller input commands, and the Windows system on which the application will focus will be utilized. Qt is a cross-platform library for developing graphical interface applications and provides everything needed for such an application.

Upon entering the program, the user is prompted to enter the username that the application will use. They then enter the main window of the application, which is primarily navigated using the mouse and the controller. This window has three tabs: one with the user's applications, one with the user's favorite applications, and one with the application's settings, where it can be customized. In the settings, there is also a button to open the virtual keyboard window, which is also accessible via the video game controller. Additionally, there is a window for executing applications via the keyboard. This window facilitates the execution of applications using this device by adding navigation functionality that addresses issues such as focusing on the wrong objects in the graphical interface.

The application's code includes a class for creating the program's graphical interface, a class for creating the virtual keyboard, and a class for controlling input commands from the video game controller. Additionally, there are functions that "simulate" keystrokes, which are used to map the controller button presses to shortcuts available in the system.

Finally, this application can be installed via an installer created for this purpose. This installer is available in the corresponding repository on GitHub and includes the source code of the project.