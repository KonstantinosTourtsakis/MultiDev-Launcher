






Η χρήση του προσωπικού υπολογιστή έχει παραμείνει η ίδια εδώ και δεκάδες χρόνια.
Σε αντίθεσή όμως με τις εποχές που οι υπολογιστές αποτελούσαν συσκευές που λίγοι
είχαν στην διάθεσή τους, πλέον υπάρχουν σε ένα μεγάλο ποσοστό στα σπίτια των
ανθρώπων. Η χρήση για την οποία προορίζεται επομένως έχει διαφοροποιηθεί και
για αυτόν τον λόγο είναι αναγκαίο να μπορεί ο χρήστης να περιηγηθεί μέσα στο σύστημα
ανάλογα με τις συσκευές εισόδου που ικανοποιούν και αιτιολογούν την χρήση αυτή.
Το πρόγραμμα αυτό προσπαθεί να λύσει αυτό το πρόβλημα, προσθέτοντας λειτουργικότητα
με την οποία ο χρήστης θα μπορεί να περιηγηθεί χρησιμοποιώντας είτε το πληκτρολόγιο,
είτε το ποντίκι είτε ένα χειριστήριο κονσόλας βιντεοπαιχνιδιών. Για την υλοποίηση του
προγράμματος αυτού θα αξιοποιηθεί η βιβλιοθήκη γραφικής διεπαφής Qt, η βιβλιοθήκη
XInput με το οποίο γίνεται διαχείριση εντολών εισόδου χειριστηρίων Xbox και το
σύστημα Windows στο οποίο θα εστιάσει η εφαρμογή. Το Qt είναι μια cross-platform βιβλιοθήκη
για ανάπτυξη εφαρμογών γραφικής διεπαφής και παρέχει οτιδήποτε χρειάζεται μια τέτοιου είδους εφαρμογή. Ο χρήστης κατά την εισαγωγή του
στο πρόγραμμα καλείται να εισάγει το όνομα χρήστη που θα χρησιμοποιήσει η εφαρμογή.
Στην συνέχεια εισέρχεται στο βασικό παράθυρο της εφαρμογής το οποίο αξιοποιείται κυρίως
από το ποντίκι και  το χειριστήριο. Το παράθυρο αυτό έχει 3 καρτέλες. Μια με τις εφαρμογές
του χρήστη, μία με τις αγαπημένες εφαρμογές του χρήστη και μία με τις ρυθμίσεις της εφαρμογής
στις οποίες μπορεί να την παραμετροποιήσει . Στις ρυθμίσεις μπορεί επίσης να βρει και ένα
κουμπί για το άνοιγμα του παραθύρου ενός εικονικού πληκτρολογίου στο οποίο έχει πρόσβαση και
μέσω του χειριστηρίου βιντεοπαιχνιδιών. Επιπλέον υπάρχει ένα παράθυρο για την εκτέλεση 
εφαρμογών μέσω του πληκτρολογίου. Αυτό το παράθυρο διευκολύνει την εκτέλεση τους μέσω της
συσκευής αυτής προσθέτοντας λειτουργικότητα περιήγησης πάνω σε αυτό που αντιμετωπίζει προβλήματα
όπως εστίαση σε λάθος αντικείμενα πάνω στην γραφική διεπαφή. Στον κώδικα της εφαρμογής υπάρχει
μια κλάση στην οποία δημιουργείται η γραφική διεπαφή του προγράμματος αυτού, μια κλάση για την
δημιουργία του εικονικού πληκτρολογίου και μια κλάση για τον έλεγχο εντολών εισόδου από το
χειριστήριο βιντεοπαιχνιδιών. Επιπλέον υπάρχουν συναρτήσεις που "προσομοιώνουν" πατήματα πλήκτρων
και χρησιμεύουν στην αντιστοίχηση των πιέσεων των κουμπιών του χειριστηρίου σε συντομεύσεις που
υπάρχουν στο σύστημα. Τέλος, η εφαρμογή αυτή μπορεί να εγκατασταθεί μέσω ενός installer που έχει
δημιουργηθεί για τον σκοπό αυτό. Ο installer αυτός είναι διαθέσιμος στο αντίστοιχο αποθετήριο που
υπάρχει στο GitHub και περιλαμβάνει τον πηγαίο κώδικα της εργασίας.